
%
% Transportation Research Board conference paper template
% modified version of the template available here:
% https://github.com/chiehrosswang/TRB_LaTeX_tex
% 
% 
% When numbered option is activated, lines are numbered.
\documentclass[numbered]{trbstyle}




% Put here what will go to headers as author
\AuthorHeaders{Smith, Johnson and Williams}
\title{Template for TRB Annual Meeting Paper}

\author{%
  \textbf{John A. Smith}\\
  jsmith@university.edu\\
  \hfill\break% this is a way to add line numbering on empty line
  \textbf{Sarah B. Johnson, Ph.D.}\\
  \hfill\break%
  \textbf{Michael C. Williams, Ph.D., Corresponding Author}\\
  mwilliams@research.org
}

% If necessary modify the number of words per table or figure default is set to
% 250 words per table
% \WordsPerTable{250}

% If words are counted manually, put that number here. This does not include
% figures and tables. This can also be used to avoid problems with texcount
% program i.e. if one does not have it installed.
% \TotalWords{200}

\begin{document}
\maketitle

\section{Abstract}

Abstract

\hfill\break%
\noindent\textit{Keywords}: Keyword1, Keyword2
\newpage

\section{Introduction}

More details about the manuscript details can be found online at
~\url{http://onlinepubs.trb.org/onlinepubs/AM/InfoForAuthors.pdf}.

This template is based on the 3.1 Lite unofficial TRB template (\url{https://github.com/chiehrosswang/TRB_LaTeX_tex}). The main differences are how it does the bibliography, and that you can choose any filename for the main .tex file (instead of having to name it trb\_template.tex).

The bibliography style is \verb1elsarticle-num-names1. You use the \verb1\citet1 and \verb1\citep1 commands like in standard \verb1natbib1.
To do an in text citation like \citet{Williams2022}, type \verb1\citet{Williams2022}1. To do parenthetical citations like \citep{Smith2023,Chen2022}, use \verb1\citep{Smith2023,Chen2022}1. The template comes with a BibFile.bib file you can put your bibtex entries in.

Here is a figure. Reference it with \verb1\ref{fig:figure}1 like Figure~\ref{fig:figure}. Likewise, you can reference a table with \verb1\ref{tab:table}1 like Table~\ref{tab:table}.

\begin{figure}[!ht]
  \centering
  \includegraphics[width=0.6\textwidth]{example-image-a}
  \caption{Put your figure here. A figure is worth 250 words.}\label{fig:figure}
\end{figure}

\begin{table}[!ht]
  \caption{Here is a table.}\label{tab:table}
  \begin{center}
  \begin{tabular}{l r r r r}
    Mode          & Passengers (M) & Distance (km) & Cost (\$) & Time (min) \\\hline
    Bus           & 245.3          & 12.4          & 2.50      & 35         \\
    Light Rail    & 189.7          & 8.9           & 3.00      & 22         \\
    Subway        & 892.1          & 15.2          & 2.75      & 28         \\
    Rideshare     & 156.4          & 9.7           & 15.50     & 18         \\
    Bike Share    & 78.2           & 3.5           & 4.00      & 25         \\
    Ferry         & 45.9           & 11.8          & 5.50      & 40         \\
    Commuter Rail & 234.8          & 42.3          & 8.75      & 55         \\
      Streetcar     & 67.5           & 5.6           & 2.25      & 30         \\\hline
    \end{tabular}
  \end{center}
\end{table}


For equations, use the \verb1\begin{linenomath}1 and \verb1\end{linenomath}1 commands. Here are three equations aligned with the flalign environment.

\begin{linenomath}
  \begin{flalign}
     & \alpha_1 + \beta_1 = \gamma_1 & \\
     & \alpha_2 + \beta_2 = \gamma_2 & \\
     & \alpha_3 + \beta_3 = \gamma_3 &
  \end{flalign}
\end{linenomath}

Here is a regular equation:

\begin{linenomath}
  \begin{equation}
    s^*(v_\alpha,\Delta v_\alpha) = s_0 + v_\alpha\,T + \frac{v_\alpha\,\Delta v_\alpha}{2\,\sqrt{a\,b}}
  \end{equation}
\end{linenomath}

\section{Acknowledgements}

This template is based on the 3.1 Lite unofficial TRB template (\url{https://github.com/chiehrosswang/TRB_LaTeX_tex}) created by Chieh Ross Wang, David Pritchard, and Gregory Macfarlane. 

\newpage

\bibliographystyle{elsarticle-num-names}
\bibliography{BibFile}
\end{document}
